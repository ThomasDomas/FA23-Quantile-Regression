% Options for packages loaded elsewhere
\PassOptionsToPackage{unicode}{hyperref}
\PassOptionsToPackage{hyphens}{url}
\PassOptionsToPackage{dvipsnames,svgnames,x11names}{xcolor}
%
\documentclass[
  letterpaper,
  DIV=11,
  numbers=noendperiod]{scrreprt}

\usepackage{amsmath,amssymb}
\usepackage{iftex}
\ifPDFTeX
  \usepackage[T1]{fontenc}
  \usepackage[utf8]{inputenc}
  \usepackage{textcomp} % provide euro and other symbols
\else % if luatex or xetex
  \usepackage{unicode-math}
  \defaultfontfeatures{Scale=MatchLowercase}
  \defaultfontfeatures[\rmfamily]{Ligatures=TeX,Scale=1}
\fi
\usepackage{lmodern}
\ifPDFTeX\else  
    % xetex/luatex font selection
\fi
% Use upquote if available, for straight quotes in verbatim environments
\IfFileExists{upquote.sty}{\usepackage{upquote}}{}
\IfFileExists{microtype.sty}{% use microtype if available
  \usepackage[]{microtype}
  \UseMicrotypeSet[protrusion]{basicmath} % disable protrusion for tt fonts
}{}
\makeatletter
\@ifundefined{KOMAClassName}{% if non-KOMA class
  \IfFileExists{parskip.sty}{%
    \usepackage{parskip}
  }{% else
    \setlength{\parindent}{0pt}
    \setlength{\parskip}{6pt plus 2pt minus 1pt}}
}{% if KOMA class
  \KOMAoptions{parskip=half}}
\makeatother
\usepackage{xcolor}
\setlength{\emergencystretch}{3em} % prevent overfull lines
\setcounter{secnumdepth}{-\maxdimen} % remove section numbering
% Make \paragraph and \subparagraph free-standing
\ifx\paragraph\undefined\else
  \let\oldparagraph\paragraph
  \renewcommand{\paragraph}[1]{\oldparagraph{#1}\mbox{}}
\fi
\ifx\subparagraph\undefined\else
  \let\oldsubparagraph\subparagraph
  \renewcommand{\subparagraph}[1]{\oldsubparagraph{#1}\mbox{}}
\fi


\providecommand{\tightlist}{%
  \setlength{\itemsep}{0pt}\setlength{\parskip}{0pt}}\usepackage{longtable,booktabs,array}
\usepackage{calc} % for calculating minipage widths
% Correct order of tables after \paragraph or \subparagraph
\usepackage{etoolbox}
\makeatletter
\patchcmd\longtable{\par}{\if@noskipsec\mbox{}\fi\par}{}{}
\makeatother
% Allow footnotes in longtable head/foot
\IfFileExists{footnotehyper.sty}{\usepackage{footnotehyper}}{\usepackage{footnote}}
\makesavenoteenv{longtable}
\usepackage{graphicx}
\makeatletter
\def\maxwidth{\ifdim\Gin@nat@width>\linewidth\linewidth\else\Gin@nat@width\fi}
\def\maxheight{\ifdim\Gin@nat@height>\textheight\textheight\else\Gin@nat@height\fi}
\makeatother
% Scale images if necessary, so that they will not overflow the page
% margins by default, and it is still possible to overwrite the defaults
% using explicit options in \includegraphics[width, height, ...]{}
\setkeys{Gin}{width=\maxwidth,height=\maxheight,keepaspectratio}
% Set default figure placement to htbp
\makeatletter
\def\fps@figure{htbp}
\makeatother

\KOMAoption{captions}{tableheading}
\makeatletter
\makeatother
\makeatletter
\makeatother
\makeatletter
\@ifpackageloaded{caption}{}{\usepackage{caption}}
\AtBeginDocument{%
\ifdefined\contentsname
  \renewcommand*\contentsname{Table of contents}
\else
  \newcommand\contentsname{Table of contents}
\fi
\ifdefined\listfigurename
  \renewcommand*\listfigurename{List of Figures}
\else
  \newcommand\listfigurename{List of Figures}
\fi
\ifdefined\listtablename
  \renewcommand*\listtablename{List of Tables}
\else
  \newcommand\listtablename{List of Tables}
\fi
\ifdefined\figurename
  \renewcommand*\figurename{Figure}
\else
  \newcommand\figurename{Figure}
\fi
\ifdefined\tablename
  \renewcommand*\tablename{Table}
\else
  \newcommand\tablename{Table}
\fi
}
\@ifpackageloaded{float}{}{\usepackage{float}}
\floatstyle{ruled}
\@ifundefined{c@chapter}{\newfloat{codelisting}{h}{lop}}{\newfloat{codelisting}{h}{lop}[chapter]}
\floatname{codelisting}{Listing}
\newcommand*\listoflistings{\listof{codelisting}{List of Listings}}
\makeatother
\makeatletter
\@ifpackageloaded{caption}{}{\usepackage{caption}}
\@ifpackageloaded{subcaption}{}{\usepackage{subcaption}}
\makeatother
\makeatletter
\@ifpackageloaded{tcolorbox}{}{\usepackage[skins,breakable]{tcolorbox}}
\makeatother
\makeatletter
\@ifundefined{shadecolor}{\definecolor{shadecolor}{rgb}{.97, .97, .97}}
\makeatother
\makeatletter
\makeatother
\makeatletter
\makeatother
\ifLuaTeX
  \usepackage{selnolig}  % disable illegal ligatures
\fi
\IfFileExists{bookmark.sty}{\usepackage{bookmark}}{\usepackage{hyperref}}
\IfFileExists{xurl.sty}{\usepackage{xurl}}{} % add URL line breaks if available
\urlstyle{same} % disable monospaced font for URLs
\hypersetup{
  colorlinks=true,
  linkcolor={blue},
  filecolor={Maroon},
  citecolor={Blue},
  urlcolor={Blue},
  pdfcreator={LaTeX via pandoc}}

\author{}
\date{}

\begin{document}
\ifdefined\Shaded\renewenvironment{Shaded}{\begin{tcolorbox}[breakable, sharp corners, borderline west={3pt}{0pt}{shadecolor}, boxrule=0pt, enhanced, frame hidden, interior hidden]}{\end{tcolorbox}}\fi

\hypertarget{methods}{%
\chapter{Methods}\label{methods}}

\hypertarget{ordinary-least-squares}{%
\section{Ordinary least squares}\label{ordinary-least-squares}}

Ordinary least squares model or OLS, works by creating a line through
the data points. Then it calculates the difference between each
prediction and observation (residual). And it tries to minimize the
squared value of the residuals. The ordinary least squares is defined
by:

\[
y_i=\alpha+\beta x_i+\varepsilon_i .
\]

The least squares estimates in this case are given by simple formulas

\[
\widehat{\beta} =\frac{\sum_{i=1}^n\left(x_i-\bar{x}\right)\left(y_i-\bar{y}\right)}{\sum_{i=1}^n\left(x_i-\bar{x}\right)^2}
\]

\[
\widehat{\alpha} =\bar{y}-\widehat{\beta} \bar{x}
\]

\hypertarget{quantile-regression}{%
\section{Quantile regression}\label{quantile-regression}}

In Koneker's 1978 paper, the \(\theta^{th}\) Quantile regression is
defined as any solution to the following problem:

\begin{equation}
\min _{b \in \mathbf{R}^K}\left[\sum_{t \in\left\{t: y_t \geqslant x_t b\right\}} \theta\left|y_t-x_t b\right|+\sum_{t \in\left\{t: y_t<x_t b\right\}}(1-\theta)\left|y_t-x_t b\right|\right] 
\end{equation}

where

\[
\{x_t: t=1,..., T\}
\] denotes a sequence (row) of K-vectors of a known design matrix and
\[\{y_t: t=1,..., T\}\] is a random sample on the regression process
\(u_t = y_t - x_t\beta\) {[}1{]}.

\hypertarget{evaluation-metrics}{%
\section{Evaluation metrics}\label{evaluation-metrics}}

\hypertarget{mean-absolute-error}{%
\subsection{Mean absolute error}\label{mean-absolute-error}}

The mean absolute error (MAE) is the average magnitude of the errors of
the values predicted by the regression and the actual observed values
for the response variable. Because it is a simple average, all errors
have the same weight, there are no penalties for different magnitude
deviations {[}2{]}. MAE assumes that the errors are normally
distributed, if the error distribution was non-normal, the average may
not be a good measure of centrality and can paint a false picture of the
goodness-of-fit of the regression curve. MAE also assumes that the
errors are unbiased. While the average magnitude of the errors is
expected to be non-zero (unless the regression is a perfect fit) the
average of the residuals, i.e., the deviation of the predicted value
from the actual value, considering underestimation and overestimation.
This means on average the regression curve does not over or
underestimate.

\[
\text { MAE }=\frac{1}{n}\sum_{i=1}^n\left|y_i-\hat{y}_i\right|=\frac{1}{n}\sum_{i=1}^n\left|e_i\right|
\]

\hypertarget{root-mean-squared-error}{%
\subsection{Root mean squared error}\label{root-mean-squared-error}}

It calculates the differences between the predictions and the actual
observations (residuals) and then gets their quadratic mean for each.
This type of error gives a larger penalty for larger errors {[}2{]}.
This error also assumes that the errors are unbiased and that they
follow a normal distribution. This gives a picture of the size of
residuals in comparison to the regression line.

\[
\operatorname{RMSE}=\sqrt{\operatorname{MSE}}=\sqrt{\frac{1}{n}\sum_{i=1}^n (y_i-\hat{y}_i)^2}=\sqrt{\frac{1}{n}\sum_{i=1}^n e_i^2}
\]

\hypertarget{variance-of-error}{%
\subsection{Variance of error}\label{variance-of-error}}

It is a measure of how spread all the errors are from the mean of all
errors. \[
\operatorname{Var}(e)=\frac{1}{n}\sum_{i=1}^n(e_i-\bar{e})^2
\]

\hypertarget{minmax-error}{%
\subsection{Min/max error}\label{minmax-error}}

A measure of the maximum residual for a prediction and the minimum
residual.

\[
f: X \rightarrow \mathbb{R} \text {, if }(\forall e \in X_{error}) f\left(e_i\right) \geq f(e)
\]

\[
f: X \rightarrow \mathbb{R} \text {, if }(\forall e \in X_{error}) f\left(e_i\right) \leq f(e)
\]

\hypertarget{anova-test}{%
\subsection{ANOVA test??}\label{anova-test}}



\end{document}
